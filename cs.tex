\documentclass[hyperref={pdfpagelabels=false}, color=table]{beamer}
\usetheme{Madrid}
\useoutertheme[subsection=true]{miniframes}
\usepackage{lmodern}
\usepackage{tikz}
\usepackage{booktabs}
\usepackage{bookmark}
%\usepackage[usenames,dvipsnames]{color}
\usepackage{pgffor}
\usepackage{csvsimple}
\usepackage{colortbl}
\usepackage[first=0,last=9]{lcg}
\definecolor{LightCyan}{rgb}{0.88,1,1}
\newcommand{\ra}{\rand0.\arabic{rand}}
\usepackage{graphicx}
\usepackage{color}
\usetheme{Madrid}
\title{Corsound Interview Practical Test}
\author{Mor Zahavi}
\date{Corsound AI}
\begin{document}
    \logo{\includegraphics[scale=0.14]{logo}}
    %   \logo{\includegraphics‏[scale=0.14]{figures/logo}}
    \begin{frame}
        \titlepage
    \end{frame}
    \begin{frame}\frametitle{Table of contents}
    \tableofcontents
    \end{frame}
    \section{Introduction}
    \subsection{Motivation}
    \begin{frame}

            The goal is to fit a classifier to distinguish real speech from fake. A solution is expected to
            be a DL model that having an audio input (waveform) outputs a score with associated
            threshold to classify real/fake speech. As a target metric we suggest using EER (equal error




            \end{frame}
    \begin{frame}
            Why is Equal Error Rate (EER) important?

            EER is important in biometric systems for several reasons.
            First, a high EER ensures that the system is able to accurately identify and verify individuals, which is critical for security and access control applications.
            A low EER can lead to false rejections or acceptances, which can compromise the integrity of the system.
            Second, the EER can provide insight into the usability and user experience of the biometric system.
            A high EER indicates that the system is user-friendly and easy to use, which can encourage users to adopt and continue using the system.
            A low EER, on the other hand, may indicate that the system is difficult to use, leading to frustration and potential abandonment.
            Finally, the EER can also be used as a benchmark for comparing different biometric systems.
            By comparing the EERs of different systems, organizations can determine which system is best suited for their specific needs and requirements.
            In summary, the EER is an important metric in biometric systems that measures the accuracy, usability, and effectiveness of the system.
            A high EER is essential for security and access control applications, while also providing insight into the user experience and facilitating system comparisons.




            𝐸𝐸𝑅=𝐹𝐴𝑅+𝐹𝑅𝑅2

    \end{frame}


    \end{document}

    \begin{frame}{Why FBprophet?}
        \begin{itemize}
            \item Popular package for time series analysis
            \item It is fully automated, tunable, fast and accurate.
            \item Provides data on most holidays and major events in many countries
            \item Varying importance to preceding and succeeding days of event
            \item Tuning change points sensitivity
        \end{itemize}
    \end{frame}
    \section{The Model}
    \begin{frame}
        \begin{equation}
            Y_{t,m} =
            X^{\ total}_{t,m} +
            G_{t,m} +
            G^{\ paid}_{t,m} +
            M_{t,m} +
            T_{t,m}
        \end{equation}
        Where:
        \footnotesize
        \begin{itemize}
            \item $Y_{t,m}$ is the total paid appointments in market $m$ on day $t$
            \item $X^{\ total}_{t,m}$ is the total appointments created in market $m$ on day $t$
            \item $G_{t,m}$ is the total Google appointments created in market $m$ on day $t$
            \item $G^{\ paid}_{t,m}$ is the total paid Google appointments created in market $m$ on day $t$
            \item $M^{\ cost}_{t,m}$ is the total marketing cost in market $m$ on day $t$
            \item $T^{\ total}_{t,m}$ is the total number of technicians who claimed a paid appointment in market $m$ on day $t$
        \end{itemize}
    \end{frame}
    \begin{frame}{Data}
        \begin{itemize}
            \item Daily data on the number of paid and unpaid appointments
            \item Series begins on Jan 2019 (\textasciitilde 1,200 rows)
            \item Controlling for Google appt, Google cost and number of active techs
            \item Controlling for holidays and major events in the US\@.
        \end{itemize}
    \end{frame}

    \section{Demo}
    \subsection{Big Markets - DC}
    \begin{frame}\frametitle{Data and forecast}
    \begin{figure}\centering
    \includegraphics[scale=.35]{/Users/mz/puls/bi-models/demsup/presentation/figures/fig_1_DC.pdf}
    \end{figure}
    \note{Black dots are real data, blue is confidence intervals,
        the blue at the right hand side show future prediction.
    }
    \end{frame}
    \begin{frame}\frametitle{Trend Components}
    \begin{figure}
        \includegraphics[scale=.25, trim={0cm 15cm 0 0cm},clip]{/Users/mz/puls/bi-models/demsup/presentation/figures/fig_2_DC}
    \end{figure}
    \end{frame}
    \begin{frame}\frametitle{Data Change-points}
    \begin{figure}
        \includegraphics[scale=.35]{/Users/mz/puls/bi-models/demsup/presentation/figures/fig_3_DC}
    \end{figure}
    \end{frame}
    \begin{frame}{Performance}
        \begin{itemize}
            \item Metrics (MAPE, sMAPE, rmse)
            \item Real world application
            \item Caveats - Market size (?)
        \end{itemize}
        \begin{center}

            \begin{minipage}{10cm}
                \begin{block}{MAPE}
                    Context dependent.
                    (Gilliland, Michael. 2010) However over 10\% is considered good, and under 20\% is risky.
                \end{block}
            \end{minipage}
        \end{center}
    \end{frame}
    \subsection{Cross Validation}
    \begin{frame}\frametitle{Diagnostics}
    \begin{figure} \centering
    \begin{tikzpicture}[
        every node/.style={anchor=south west,inner sep=0pt},
        x=1mm, y=1mm,]
        \node
        (rect) at (42,-.5) [draw,color=red,minimum width=.7cm,minimum height=4.7cm]
            { };
        \node (fig2) at (0,0)
            {            \resizebox{.7\textwidth}{!}{
            \csvautotabular{df_p_dc.csv}
        }};
    \end{tikzpicture}
    \end{figure}
    \end{frame}
    \begin{frame}\frametitle{MAPE}
    \begin{figure}
        \includegraphics[scale=.2]{/Users/mz/puls/bi-models/demsup/presentation/figures/fig_4_DC}
    \end{figure}
    \end{frame}
    \section{Mid Size Markets - Tampa}
    \begin{frame}\frametitle{Other Markets}
    \begin{figure} \centering
    \begin{tikzpicture}[
        every node/.style={anchor=south west,inner sep=0pt},
        x=1mm, y=1mm,
    ]
        \node (fig2) at (0,0)
            {        \includegraphics[scale=.3]{figures/markets}};
        \node
        (rect) at (25.2,4) [draw,color=red,minimum width=2.2cm,minimum height=1.3cm]
            { };
    \end{tikzpicture}
    \end{figure}
    \end{frame}
    \begin{frame}\frametitle{Data and forecast}
    \begin{figure}\centering
    \includegraphics[scale=.35]{/Users/mz/puls/bi-models/demsup/presentation/figures/fig_1_Tampa}
    \end{figure}
    \note{Black dots are real data, blue is confidence intervals,
        the blue at the right hand side show future prediction.
    }
    \end{frame}
    \begin{frame}\frametitle{Trend Components}
    \begin{figure}
        \includegraphics[scale=.25, trim={0cm 15cm 0 0cm},clip]{/Users/mz/puls/bi-models/demsup/presentation/figures/fig_2_Tampa}
    \end{figure}
    \end{frame}
    \begin{frame}\frametitle{Data Change-points}
    \begin{figure}
        \includegraphics[scale=.35]{/Users/mz/puls/bi-models/demsup/presentation/figures/fig_3_Tampa}
    \end{figure}
    \end{frame}
    \begin{frame}\frametitle{Diagnostics}
    \begin{figure} \centering
    \begin{tikzpicture}[
        every node/.style={anchor=south west,inner sep=0pt},
        x=1mm, y=1mm,
    ]
        \node
        (rect) at (42,-.5) [draw,color=red,minimum width=.7cm,minimum height=4.7cm]
            { };
        \node (fig2) at (0,0)
            {            \resizebox{.7\textwidth}{!}{
            \csvautotabular{df_p_tampa.csv}
        }};
    \end{tikzpicture}
    \end{figure}
    \end{frame}
    \begin{frame}\frametitle{MAPE}
    \begin{columns}
        \column{0.5\textwidth}
        \centering
        \begin{figure}
            \includegraphics[scale=.13]{/Users/mz/puls/bi-models/demsup/presentation/figures/fig_4_Tampa}
        \end{figure}
        \column{0.5\textwidth}
        \centering
        \begin{figure}
            \includegraphics[scale=.13]{/Users/mz/puls/bi-models/demsup/presentation/figures/fig_4_DC}
        \end{figure}
    \end{columns}
    \end{frame}
    \section{Small Markets - Denver}
    \begin{frame}\frametitle{Denver}
    \begin{figure} \centering
    \begin{tikzpicture}[
        every node/.style={anchor=south west,inner sep=0pt},
        x=1mm, y=1mm,
    ]
        \node (fig2) at (0,0)
            {        \includegraphics[scale=.3]{figures/markets}};
        \node
        (rect) at (25.2,4) [draw,color=red,minimum width=2.2cm,minimum height=1.3cm]
            { };
    \end{tikzpicture}
    \end{figure}
    \end{frame}
    \begin{frame}\frametitle{Data and forecast}
    \begin{figure}\centering
    \includegraphics[scale=.35]{/Users/mz/puls/bi-models/demsup/presentation/figures/fig_1_Denver}
    \end{figure}
    \note{Black dots are real data, blue is confidence intervals,
        the blue at the right hand side show future prediction.
    }
    \end{frame}
    \begin{frame}\frametitle{Trend Components}
    \begin{figure}
        \includegraphics[scale=.25, trim={0cm 15cm 0 0cm},clip]{/Users/mz/puls/bi-models/demsup/presentation/figures/fig_2_Denver}
    \end{figure}
    \end{frame}
    \begin{frame}\frametitle{Data Change-points}
    \begin{figure}
        \includegraphics[scale=.35]{/Users/mz/puls/bi-models/demsup/presentation/figures/fig_3_Denver}
    \end{figure}
    \end{frame}
    \begin{frame}\frametitle{Diagnostics}
    \begin{figure} \centering
    \begin{tikzpicture}[
        every node/.style={anchor=south west,inner sep=0pt},
        x=1mm, y=1mm,
    ]
        \node
        (rect) at (42,-.5) [draw,color=red,minimum width=.7cm,minimum height=4.7cm]
            { };
        \node (fig2) at (0,0)
            {            \resizebox{.7\textwidth}{!}{
            \csvautotabular{df_p_denver.csv}
        }};
    \end{tikzpicture}
    \end{figure}
    \end{frame}
    \begin{frame}\frametitle{MAPE}
    \begin{columns}
        \column{0.5\textwidth}
        \centering
        \begin{figure}
            \includegraphics[scale=.13]{/Users/mz/puls/bi-models/demsup/presentation/figures/fig_4_Denver}
        \end{figure}
        \column{0.5\textwidth}
        \centering
        \begin{figure}
            \includegraphics[scale=.13]{/Users/mz/puls/bi-models/demsup/presentation/figures/fig_4_Tampa}
        \end{figure}
    \end{columns}
    \end{frame}
    \begin{frame}{MAPE - 7 days prediction comparison}
        \begin{itemize}
            \item DC - 0.04
            \item Tampa - 0.12
            \item Denver - 0.2
            \item Newark 0.11
        \end{itemize}
    \end{frame}

%
%
    \foreach \n in {1,...,4}{
        \begin{frame}\frametitle{Newark \n}
        \begin{figure}\centering
        \includegraphics[scale=.2]{fig_\n_newark.pdf}
        \end{figure}
        \end{frame}
    }
    \begin{frame}
        \begin{figure} \centering
        \begin{tikzpicture}[
            every node/.style={anchor=south west,inner sep=0pt},
            x=1mm, y=1mm,
        ]
            \node
            (rect) at (42,-.5) [draw,color=red,minimum width=.7cm,minimum height=4.7cm]
                { };
            \node (fig2) at (0,0)
                {            \resizebox{.7\textwidth}{!}{
                \csvautotabular{df_p_newark.csv}
            }};
        \end{tikzpicture}
        \end{figure}
    \end{frame}

\end{document}


































